\begin{newdef}
	设$X$为一非空几何,如果对于$X$中任给的两个\textbf{元素}$x,\ y$,均有一个确定的实数,记为$\rho\left( x, \ y \right)$,与他们对应且满足下面三个条件:\\
	(i) \textbf{非负性:}$\rho (x,\ y)\geqslant 0, \rho (x,\ y)=0 $的充分必要条件是$x=y$;\\
	(ii) \textbf{对称性:}$\rho (x,\ y)=\rho (y, \ x)$\\
	(iii) \textbf{三角不等式:} $\rho (x,\ y)\leqslant \rho (x,\ z)+\rho (z, \ y)$,这里$z$也是$X$中任意一个元素.\\
	则称$\rho $是$X$上的一个\textbf{距离},而称$X$是以$\rho$为距离的距离空间,记为$(X, \rho)$\\
	设$X$为一非空几何,如果对于$X$中任给的两个\textbf{元素}$x,\ y$,均有一个确定的实数,记为$\rho\left( x, \ y \right)$,与他们对应且满足下面三个条件:
\end{newdef}

\begin{note}
	设$X$,定义距离$\rho(x, \ x )=0,\  \rho(x, \ y)=1 , \ x\neq y, \ x, \ y \in X$,则称$(X,\rho)$为离散的距离空间
\end{note}


\begin{newex}
	\textbf{$n$维欧式空间 $\mathbb R ^n$}
	其上的距离为:
	\begin{equation}
		\rho (x, \ y)=\big( \sum\limits_{k=1}^n | \xi_k- \eta_k|^2 \big)^{1/2} \label{eq:rho1}
	\end{equation}
\end{newex}

\begin{note}
	1.此处证明使用了柯西(A.Cauthy)不等式
	\begin{align}
		\big(\sum \limits_{k=1}^n a_k b_k \big)^2 \leqslant \sum \limits_{k=1}^n a_k^2 \cdot \sum \limits_{k=1}^n b_k^2
	\end{align}
	2.另外可以引入距离
	\begin{align*}
		\rho_1(x,\ y)=\max \limits_{1\leqslant k \leqslant n}|\xi_k-\eta_k|. \tag*{(1')}
		\label{eq:rho1'}
	\end{align*}
	3.对于$\mathbb C^n$同样可以用$\rho$定义距离空间
\end{note}

\begin{newex}
	空间$C[a,\ b]$
	\begin{align}
		\rho(x, \ y)=\max\limits_{a \leqslant t \leqslant b} |x(t)-y(t)|
		\label{eq:rho3}
	\end{align}
\end{newex}

\begin{newex}
	$L^p(F) \ (1 \leqslant p < \infty , \ F \subset \mathbb R , \mbox{且为可测集})$\
	\begin{align}
		\rho (x , \ y)= \big( \int_{F} |x(t)-y(t)|^p \dif t \big)^{1/p}
		\label{eq:rho4}
	\end{align}
\end{newex}
\begin{note}
	两个几乎处处相等的$p$幂可积函数在$L^p(F)$中视为同一元素.
\end{note}

\begin{newex}
	空间$L^p (F)$
	\begin{align}
		\rho (x, \ y) & = \inf _{\underset{  F_0 \subset F}{  \mathrm{m} F_0=0}}
		\{ \sup\limits_{t \in F \backslash F_0}  | x(t)-y(t)|  \} \notag         \\
		              & = \underset{t \in F }{\mathrm{esssup}} |x(t) - y(t)|
	\end{align}
\end{newex}
\begin{note}
	1. 称定义在可测集$F$上的可测函数$x(\ \cdot \ )$是\textbf{本性有界}的,如果存在$F$
	的某个零测度子集$F_0$,使得$x( \ \cdot \ )$ 在集合$F\backslash F_0$上有界.\\
	2. $F$上所有本性有界可测函数构成的集合用$L^\infty (F)$表示.\\
	3. 几乎处处相等的两个本性有界的可测函数看做同一元素.
	4. 证明其中第三条性质的时候. 用好sup的定义
\end{note}

\begin{newex}
	空间$l^p (1 \leqslant p < \infty)$
	令$l^p$是满足下列不等式的实(或复)数列$x=\{ \xi_1,\ \xi_2, \ \cdots, \ \xi_n,\ \cdots  \}$
	构成的集合:$$\sum \limits_{n=1}^\infty |\xi_n|^p<\infty$$
	\begin{align}
		\rho(x, \ y)=\big(\ \sum \limits_{n=1}^\infty |\xi_n-\eta_n|^p  \big)^{1/p}
	\end{align}
\end{newex}
\begin{note}
	借助以下三个不等式可得:\\
	1. Young 不等式
	\begin{align}
		u^{\frac{1}{p}}v^{ \frac{1}{q}}\leqslant \dfrac{u}{p}+\dfrac{v}{q}
	\end{align}
	2. 赫尔德不等式
	\begin{align}
		\sum \limits_{n=1}^\infty |\xi_n \eta_n| \leqslant
		\big(\ \sum \limits_{n=1}^\infty |\xi_n|^p  \big)^{1/p}
		\big(\ \sum \limits_{n=1}^\infty |\eta_n|^q  \big)^{1/q}
	\end{align}
	3. H.Minkowski不等式
	\begin{align}
		\big(\ \sum \limits_{n=1}^\infty |\xi_n+\eta_n|^p  \big)^{1/p}\leqslant
		\big(\ \sum \limits_{n=1}^\infty |\xi_n|^p  \big)^{1/p}+
		\big(\ \sum \limits_{n=1}^\infty |\eta_n|^p  \big)^{1/p}
	\end{align}
\end{note}

\begin{newex}
	空间$l^\infty$
	\begin{align}
		\rho(x, \ y)= \sup_{1 \leqslant n < \infty} |\xi_n-\eta_n|
	\end{align}
\end{newex}

\begin{newdef}
	设$\{ x_n \}$为距离空间$X$中的一个\textbf{点列}(或称\textbf{序列}),这里
	$n=1,\ 2,\ 3,\ \dots$.如果存在$X$中的点$x_0$使得当$x \rightarrow \infty$时,
	$\rho(x_n,x_0) \rightarrow 0 (n \rightarrow \infty)$.则称点列$\{x_n \}$\textbf{收敛}于$x_0$,记为
	\begin{align*}
		\lim_{n \rightarrow \infty} x_n =x_0 \quad \mbox{或} \{x_n\} \rightarrow x_0 \quad (n \rightarrow \infty)
	\end{align*}
\end{newdef}

\begin{newthem}
	设 $\{ x_n\}$是距离空间$X$中的收敛点列,则下列性质成立:\\
	(i) $\{x_n \}$的极限唯一;\\
	(ii) 对任意的$y_0 \in X$,数列$\{\rho (x_n, \ y_0) \}$有界.
\end{newthem}

\begin{newthem}
	设$\{ x_n \}$是距离空间$X$中的点列,且收敛,则$\{x_n \}$的任一子列$\{ x_{n_k} \}$
	也收敛,且收敛于同一极限.\\
	反之,若$\{ x_n \}$的任一子列收敛,则$\{ x_n\}$本身也收敛。
\end{newthem}

\begin{note}
	1. $\mathbb R^n$中的收敛,其中的点列$\{ x^{(m)}\} = \{  (\xi_1^{(m)},\ \xi_2^{(m)},\ \cdots , \ \xi_n^{(m)}) \}$
	按照等式(\eqref{eq:rho1})或按照等式(\ref{eq:rho1'})定义的距离收敛于$x = (\xi_1, \ \xi_2 , \ \cdots , \ \xi_n)$
	的充分必要条件均为$x ^{(m)}$的每个坐标收敛于$x$的相应坐标。
	2. 对于$C[a,b]$,按照距离(\ref{eq:rho3})其中点列$\{ x_n\}$收敛于点$x_0$的充分必要条件是
	:作为函数列的$\{x_n(\  \cdot\  )\}$在$\left[ a,b \right]$上一致收敛于函数$x_0(\ \cdot \ )$。
	3. 在$C[a,b]$中,并不是所有距离下的收敛都和一致收敛等价,如定义
	$$\rho_1(x,\ y)=\big(  \int_a^b |x(t)-y(t)|^2 \dif d t   \big)^{1/2}$$
	取函数列$$x_n(t)=\dfrac{1}{(b-a)^n}(t-a)^n \quad \left(  t \in \left[a,\ b \right],\ n=1,\ 2, \cdots \right) $$
	可知道,按照$\rho_1$,其收敛于0,而显然,它并不是一致收敛的(其在1附近总不一致收敛)
\end{note}

\begin{note}
	\begin{itemize}
		\item [(i)] 对于任何一个非空集合,我们都可以定义距离。但是一般来说,我们应当根据该集合的特点适当地引进距离以充分反映这些
		      特点。例如,对$C[a,\ b]$,我们常常用等式(\ref{eq:rho3})定义距离;对于$L^p(F)$我们常用等式(\ref{eq:rho4})定义距离,等等。只有这样,在理论和实践上才有意义。
		\item [(ii)] 定义距离的方式一般不说唯一。
		\item [(iii)] 如果一个非空集合中定义了两个或两个以上的距离,那么由它们导出的收敛可以等价也可以不等价。当不等价时,便得到本质
		      上不同的两个或两个以上的距离空间。
	\end{itemize}
\end{note}
