\begin{newdef}
    设$X$为一非空几何,如果对于$X$中任给的两个\textbf{元素}$x,\ y$,均有一个确定的实数,记为$\rho\left( x, \ y \right)$,与他们对应且满足下面三个条件:\\
    (i) \textbf{非负性:}$\rho (x,\ y)\geqslant 0, \rho (x,\ y)=0 $的充分必要条件是$x=y$;\\
    (ii) \textbf{对称性:}$\rho (x,\ y)=\rho (y, \ x)$\\
    (iii) \textbf{三角不等式:} $\rho (x,\ y)\leqslant \rho (x,\ z)+\rho (z, \ y)$,这里$z$也是$X$中任意一个元素.\\
    则称$\rho $是$X$上的一个\textbf{距离},而称$X$是以$\rho$为距离的距离空间,记为$(X, \ \rho)$\\
    设$X$为一非空几何,如果对于$X$中任给的两个\textbf{元素}$x,\ y$,均有一个确定的实数,记为$\rho\left( x, \ y \right)$,与他们对应且满足下面三个条件:
\end{newdef}

\begin{note}
    设$X$,定义距离$\rho(x, \ x )=0, \rho(x, \ y)=1 x\neq y, \ x, \ y \in X$,则称$(X,\rho)$为离散的距离空间
\end{note}


\begin{newex}
     \textbf{$n$维欧式空间 $\mathbb R ^n$}
    其上的距离为:
    \begin{align}
        \rho (x, \ y)=\big( \sum\limits_{k=1}^n | \xi_k- \eta_k|^2 \big)^{1/2}
    \end{align}
\end{newex}

\begin{note}
   1.此处证明使用了柯西(A.Cauthy)不等式
    \begin{align}
    \big(\sum \limits_{k=1}^n a_k b_k \big)^2 \leqslant \sum \limits_{k=1}^n a_k^2 \cdot \sum \limits_{k=1}^n b_k^2 
    \end{align}
   2.另外可以引入距离
    \begin{align*}
        \rho_1(x,\ y)=\max \limits_{1\leqslant k \leqslant n}|\xi_k-\eta_k|. \tag*{(1')}
    \end{align*}
   3.对于$\mathbb C^n$同样可以用$\rho$定义距离空间
\end{note}

\begin{newex}
    空间$C[a,\ b]$\ 
    \begin{align}
        \rho(x, \ y)=\max\limits_{a \leqslant t \leqslant b} |x(t)-y(t)|
    \end{align}
\end{newex}

\begin{newex}
    $L^p(F) \ (1 \leqslant p < \infty , \ F \subset \mathbb R , \mbox{且为可测集})$\
   \begin{align}
         \rho (x , \ y)= \big( \int_{F} |x(t)-y(t)|^p \dif t \big)^{1/p}
    \end{align} 
\end{newex}
\begin{note}
  
\end{note}
