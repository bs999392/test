\begin{newdef}
    距离空间
    \begin{align}
        \{x \colon \rho( x,\ x_0) \ <r\} \quad (r >0)
    \end{align}
    称为以$x_0$为中心,以$r$为半径的开球,也可表示为$S(x_0, \ r)$,也叫做$x_0$的一个球形邻域,或简称邻域。\
    距离空间
    \begin{align*}
        \{x \colon \rho( x,\ x_0) \ \leqslant  r\} \quad (r >0)
    \end{align*}
    称为以$x_0$为中心,以$r$为半径的闭球,也可表示为$\bar{S}(x_0, \ r)$,也叫做$x_0$的一个球形邻域,或简称邻域。
\end{newdef}

\begin{newdef}
    设$X$是一个距离空间,$G \subset X,\ x\in G.$如果存在$x$的某个邻域$S(x,\ r)\subset G$,则称$x$是$G$的内点。\ 
    $G$的全部内点构成的集合记为$G^0$,称为$G$的内部.\
    如果$G$中的每一个点都是它的内点,则称$G$为开集,空集规定为开集。
\end{newdef}
\begin{note}
    \begin{itemize}
        \item [1.] 如果$x$集合$A\subset X (A \mbox{未必是开集})$的内点,我们也称$A$是$x$的一个邻域。
    \end{itemize}
\end{note}

\begin{newthem}
    设$x$是距离空间,则\ 
    \begin{itemize}
        \item [(i)]空间$X$与空集$\emptyset$都是开集;
        \item [(ii)]任意多个开集的并是开集;
        \item [(iii)]有限多个开集的交是开集
    \end{itemize}
\end{newthem}

\begin{newdef}
    设$X$是距离空间,$A \subset X$。点$x_0 \in X$。弱队任给的$\epsilon >0 $,$x_0$的邻域$S(x_0,\ \epsilon)$中含有
    $A$中异于$x_0$的点,即$$S(x_0,\ \epsilon) \cap (A \backslash \{ x_0\}) \neq \emptyset$$
    则称$x_0$是$A$的聚点或极限点。\ 
    如果$x_0 \in A$但不是$A$的聚点,则称$x_0$为$A$的孤立点。\ 
    集合$A$的全部聚点构成的集合称为$A$的导集。记为$A'$\ 
    并集$A \cup A'$称为$A$的闭包,记为$\bar{A}$。如果$A=\bar{A} $,则称$A$为闭集。
\end{newdef}

\begin{newex}
    \begin{itemize}
        \item [1.] 设$A=\big\{1,\ \dfrac{1}{2},\ \cdots ,\ \dfrac{1}{n}   \big\}$距离为$\rho (x,\ y)=|x-y|$。0是聚点,但不在$A$中。
        \item [2.] 设$B$是区间$\left( 0,\ 1  \right]$,距离为$\rho (x,\ y)=|x-y|$,0是聚点,但不在其中。
    \end{itemize}
    因此聚点可以是内点,也可以不是内点。
\end{newex}

\begin{newex}
    设$A=\left\{  0,\ 1,\ 2,\ \cdots ,\ n,\ cdots  \right\}$,$\rho(x,\ y)=|x-y|$\ 
    $A$中的一切点都是它的孤立点,同时也是内点,内点和孤立点并不互斥。
\end{newex}

\begin{newthem}
    \begin{itemize}
        \item [(i)] $A \subset \bar{A}$
        \item [(ii)] $\bar{\bar{A}}=\bar{A}$
        \item [(iii)] $\overline{A \cup B}=\bar{A} \cup \bar{B}$
        \item [(iv)] $\bar{\emptyset }= \emptyset$ 
    \end{itemize}
\end{newthem}

\begin{newthem}
    设$X$为距离空间,则
    \begin{itemize}
        \item [(i)]空间$X$及空集$\emptyset$都是闭集;
        \item [(ii)] 任意多个闭集的交是闭集;
        \item [(iii)] 有限多个闭集的并是闭集。
    \end{itemize}
\end{newthem}

\begin{newex}
    设$X$为一离散的距离空间,则$X$中的每个点既为它的内点,也是它的孤立点,因此$X$的每个子集都是既开又闭的集。
\end{newex}

\begin{newdef}
    设$A,\ B$均为距离空间$X$的子集,如果$\bar{B} \supset A$,则称$B$在$A$中稠密.\\
    其有等价命题如下:
    \begin{itemize}
        \item [(i)] 对于任给的$x \in A$以及任给的$\varepsilon >0$,存在$B$中的点$y$使$\rho (x, \ y) <\varepsilon$
        \item [(ii)] 对于任给的$\varepsilon >0$,以$B$中的每个点为中心,以$\varepsilon$为半径的全部开球的并包含$A$
        \item [(iii)] 对于任给的$x \in A$,存在$B$中的点列$ \left\{ x_n \right\}$收敛于$x$
    \end{itemize}
\end{newdef}

\begin{note}
    在稠密集的定义中,并不要求$ B \subset A$,$B$和$A$甚至可以没有交集
\end{note}

\begin{newdef}
    设$X$为距离空间,若$X$存在稠密的可列子集,则称$X$可分.
\end{newdef}

\begin{newex}
    \begin{itemize}
        \item [(i)] $\mathbb R^n$是可分的,因为$\mathbb{ R}^n$中的坐标为有理数的点构成的集是$\mathbb{R}^n$的一个可列稠密子集
        \item [(ii)] 空间$C[a , \ b]$是可分的(因为可以利用伯恩斯坦定理证明:有理系数的多项式构成的集$P_n$在$C[a,\  b]$中稠密,而$P_0$是可列集
        \item [(iii)] 空间$L^p(F)\ (p \geqslant 1)$ 是可分的。
    \end{itemize}
\end{newex}

\begin{newex}
    $L^\infty \left[a, \ b\right]$是不可分的距离空间。设$A$是由如下的函数
   \begin{equation*}
       x_s(t)=
       \begin{cases}
           1, & a\leqslant t \leqslant s.\\
           0,  & s < t \leqslant b
       \end{cases} \qquad
       (s \in \left[a,\ b\right])
 \end{equation*}
\end{newex}

\begin{newdef}
    设$X,\ Y$都是距离空间,距离分别为$\rho, \ \rho_1$。如果对每一个$x \in X$,按照某个规律必有$Y$中唯一的$y$与之相对应,
    则称这个对应是一个\textbf{映射}。映射常用记号$T$来表示。据此,我们有$Tx=y$。\\
    如果对于某一给定的点$x_0 \in X$,映射$T$满足下面的条件:对任给的$ \varepsilon >0$,存在$\delta >0$,使得当$\rho (x,\ x_0)<\delta$
    时,有$\rho_1(Tx,\ Tx_0)<\varepsilon$,则称映射$T$在点$x_0$处\textbf{连续}。如果映射$T$在$X$中的每一点处都连续,
    则称$T$在\textbf{$X$上连续},且称$T$是\textbf{连续映射}
\end{newdef}

\begin{newdef}
    设$T$是由距离空间$X$到距离空间$Y$的映射,$A \subset X$。称集合
    $$\left\{ Tx \colon x \in A \right\} $$
    为集合$A$的\textbf{像},记为$T(A)$。设$B \subset Y$,则称集合$$\left\{ x \colon Tx \in B \right\}$$
    为集合$B$的原像,记为$T^{-1}(B)$。
\end{newdef}

\begin{newex}
    设$X$是距离空间,以$\rho$为距离,$x_0 \in X$为任一给定点,则$f(x)=\rho (x,\ x_0)$是$X$到$\mathbb R$的连续映射
    将所有由距离空间$X$到实(或复数域)的映射称为\textbf{泛函},
\end{newex}

\begin{newthem}
    距离空间$X$到距离空间$Y$中的映射$T$在点$x_0 \in X$连续的充分必要条件是对任何收敛于$x_0$的点列$\left\{x_n\right\}\subset X$,
    有$\left\{Tx_n\right\}$收敛于$Tx_0$
\end{newthem}
\begin{note}
    \begin{itemize}
        \item [(i)]其将点处$T$收敛$\Leftrightarrow$考察收敛$x_0$点列像的收敛
    \end{itemize}
\end{note}
\begin{newthem}
    距离空间$X$到距离空间$Y$中的映射$T$连续的充分必要条件是下列两个条件之一成立:
    \begin{itemize}
        \item [(i)] 对于$Y$中的任一开集$G$,$G$的原像$T^{-1} (G)$是$X$中的开集
        \item [(ii)] 对于$Y$中的任一闭集$F$,$G$的原像$T^{-1} (F)$是$X$中的闭集
    \end{itemize}
\end{newthem}
\begin{note}
    \begin{itemize}
        \item [(i)] (ii)和(i)可从余集的角度直接证明
        \item [(ii)] 证明中,可运用连续定义,构造合适的开球,包含点的像
    \end{itemize}
\end{note}

\begin{newdef}
    设$T$是由距离空间$X$到距离空间$Y$的单映射,即对任意的$x_1,\ x_2 \in X$,当$x_1 \neq x_2$时,$Tx_1 \neq Tx_2$。\\
    今再设$T$为满映射,即$T(x)=Y$,于是对任一$y \in Y$,必存在唯一的$x \in X$使得
    \begin{align}
        Tx=y
        \label{eq:6}
    \end{align}
    因此通过$\eqref{eq:6}$我们得到一个新的映射,它将$y$映成$x$,称这个映射为$T$的\textbf{逆映射},记为$$T^{-1}y=x$$
    当$T$存在逆映射时,称$T$是\textbf{可逆的}。\\
    因此,$T$存在逆映射的充分必要条件是$T$既为单映射又为满映射。它统称既单又满的映射为双映射。单映射又称为一对一映射
\end{newdef}
\begin{newdef}
    设$X,\ Y$为距离空间,距离分别为$\rho ,\ \rho_1$,又设$T$是由$X$到$Y$的映射。若$T$存在逆映射,且$T$及其逆映射$T^{-1}$
    均连续,则称$T$是$X$到$Y$上的\textbf{同胚映射}.如果存在一个从$X$到$Y$上的同胚映射,则称$X$与$Y$同胚。\\
    设$T$是双映射,且对任意的$x,\ y \in X$,有$\rho_1(Tx, \ Ty)=\rho(x,\ y)$,则称$T$是$X$到$Y$上的\textbf{等距映射}。
    如果存在一个从$X$到$Y$上的等距映射,则称$X$与$Y$\textbf{等距}。
\end{newdef}
\begin{note}
    两个同胚或等距的距离空间在很多情况下可视为同一。
\end{note}
\begin{newex}
    $y=\arctan x$是$\mathbb R$到$\big( -\dfrac{\pi}{2} , \ \dfrac{\pi}{2}\big)$上的同胚映射 \\
    $y=e^x$是$\mathbb R$到$(0,\ \infty)$上的同胚映射
\end{newex}
\begin{note}
    \begin{itemize}
        \item [(i)] 对于非空集合$X$的任意性以及在$X$上定义距离的多样性导致了距离空间的复杂性
        \item [(ii)] 注意一般距离空间与$\mathbb R^n$相似的一面,更要注意不同的另一面。最重要的是,所有内容都离不开预先给定的距离。
        \item [(iii)] 因此,如果在一个非空集合中定义了两个或两个以上的距离,应如实地将它们看成两个或两个以上不同的距离空间。
        由于它们的不同,故其中一个很可能是可分的,而其他的则不是。对于稠密性以及映射的连续性和同胚等,也类似。
    \end{itemize}    
\end{note}